%=====================================================================================
% - Title: latex
% - Author: Jelle Langedijk
% - Filename: `latex.tex`
% - Purpose: A complete, clear
%	template for writing LaTeX
%	files that is TeXlive-compliant.
%=====================================================================================
%% PREAMBLE --------------------------------------------------------------------------
% --> Document Variables
\newcommand{\Title}{My \LaTeXe{} template documentation}	%Document title
\newcommand{\Author}{J. Langedijk}						%Document author
\newcommand{\Date}{\today}								%Document date
\newcommand{\Institute}{\href{https://github.com/JelleLa/Jelles-LaTeX-Template}{Github Repository}}						%Document institute
\newcommand{\Docsummary}{								%Document summary
The features of my custom \LaTeXe template are documented here for reference.
}
% --> Setting Document Class, Input, Language and Time
\documentclass[leqno]{article}			%`leqno` places equation tags to the left
\usepackage[utf8]{inputenc}				%Unicode Support
\usepackage[english]{babel}				%Language
\usepackage[
			long,						%Long format date
			nodayofweek,				%Don't show day of week
			level,						%No clue
			24hr						%24h format
			]{datetime}					%Set date settings
\newcommand{\now}{\today, \currenttime} %Print date+time
% --> Fonts
\usepackage{kpfonts}					%Font, supported by TeXlive
\usepackage[T1]{fontenc}				%Magic thing to get the font working
% --> Header and Footer Styles
\pagestyle{headings}					%Headings page number style
\usepackage{lastpage}					%Last page
% --> Multicolumns
\usepackage{multicol}					%Multi-Columns
% --> TiKz Things
\usepackage{tikz}						%TiKz
\usepackage{pdfpages}					%PDF importer, at the wrong spot
\usetikzlibrary{positioning}			%Positioning TiKz Library
% --> Table Things
\usepackage{tabularx}					%TabularX, to make more sexy tables
\usepackage{booktabs}					%Booktabs in tables
% --> Custom Itemize
\usepackage{enumitem}					%Custom itemize behavior
\setitemize{noitemsep,
topsep=0pt,parsep=0pt,partopsep=0pt}
% --> AMSTeX Things
\usepackage{amsmath}					%All the AMSMath things
\usepackage{amssymb}					%Cute symbols
\usepackage{amsthm}						%Theorems
% --> Graphics and Figures
\usepackage{xcolor}						%Fancy colors
\usepackage{geometry}				    %Page geometry settings
\usepackage{graphicx}					%Graphics support
\usepackage{float}						%Float support
\usepackage{caption}					%For subcaptions in subfigures
\usepackage{hologo}						%Fancy Logos
% --> Graphs and Plots
\usepackage{pgfplots}					%Plotting
\pgfplotsset{							%Defaults
	compat=1.18,						%Compatibility
	clip=false,							%Do not clip plots
	width=0.9\textwidth,				%Plot width
	height=0.7\textwidth,				%Plot height
	axis lines=middle,					%Axis style
	xlabel=$x$,							%xlabel
	ylabel=$y$,							%ylabel
	grid style=dashed					%Grid style
	}
\newenvironment{plot}[2]				%Plot Environment
	{
	\newcommand{\plotcap}{#1}			%Caption variable
	\newcommand{\plotlab}{#2}			%Label variable
	\begin{figure}[H]					%Float figure
		\centering						%Center figure
		\begin{tikzpicture}				%Start TiKz environment
		\begin{axis}					%Start Axis environment
	}
	{
		\end{axis}						%Close Axis environment
		\end{tikzpicture}				%Close TiKz environment
	\caption{\plotcap}					%Figure caption
	\label{\plotlab}					%Figure label
	\end{figure}						%Close label
	}
% --> Referencing and `hyperref` Settings
\usepackage{hyperref}					%For cross-referencing, links, forms and more
\hypersetup{
    colorlinks=true,					%Links have colors
    linkcolor=cyan,						%Link color
    filecolor=magenta,					%File Color
    urlcolor=cyan,						%URL Color
	pdftitle={\Title, \Author},			%Embedded PDF Title
	pdfauthor={\Author},     			%Embedded PDF Author
    pdfsubject={\Title},   				%Embedded PDF Subject
    }
\renewcommand\LayoutCheckField[2]{#1\hfill #2}
% --> Listings and Code
\usepackage{listings}					%Code Listings
\lstdefinestyle{mystyle}{
	backgroundcolor= {},				%Transparent background
	commentstyle=\color{green},			%Comment color
	keywordstyle=\color{magenta},		%Keyword color
	numberstyle=\tiny\color{black},		%Line number color
	stringstyle=\color{purple},			%String color
	basicstyle=\ttfamily\small			%->
	\color{black},						%Font style
	breakatwhitespace=false,			%Do not break at whitespace
	breaklines=true,					%Break lines
	captionpos=b,						%Add caption below
	keepspaces=true,					%Keep spaces in code
	numbers=left,						%Line number position
	numbersep=5pt,						%Line number separation
	showspaces=false,					%Show space indicator
	showstringspaces=false,				%Show space indicator in strings
	showtabs=true,						%Show tab indicator
	tabsize=2,							%Tab size
}
\lstset{style=mystyle}					%Set `mystyle` as default listing style
% --> Tcolorbox and Code Environment
\usepackage{tcolorbox}					%Fancy Boxes around text, e.g. for code blocks
\tcbuselibrary{							%Set libraries for `tcolorbox`
	skins,								%Skins
	breakable,							%Breakable boxes
	listings}							%Listing boxes
\tcbset{listing engine=listings}		%Set listing engine
\newtcblisting[
	auto counter,
	number within=section]{code}[3]{	%Code environment
	listing only,						%Only show listing
	fonttitle=\bfseries,				%Title font
	colback=black!5!white,				%Background color
	colframe=blue!20!black,				%Frame color
	title=Listing \thetcbcounter: #1,	%Box title
	size=normal,						%Box size
	bottom=0.1 mm,						%Bottom sep.
	sharp corners,						%Use sharp corners
	breakable,							%Allow box to break
	label=#3,							%Label for referencing
	listing options={					%Options for the `listing` package
		language=#2						%Programming Language
	}}
% --> Emoji Support (Needs Compilation via LuaLaTex!)
\usepackage{emoji}						%Emoji Support
\setemojifont{TwemojiMozilla}			%Use Mozilla Emojis
% --> Appendices (\begin{<appendices})
\usepackage[toc,page]{appendix}			%Appendix usepackage
% --> BibTeX (\bibliography{<bib.tex>})
\usepackage[nottoc]{tocbibind}			%Make bibliography play nice w. TOC
\bibliographystyle{IEEEtran}			%Reference style
% --> Miscallaneous
\usepackage{lipsum}						%Generates Lorem Ipsums on demand
\usepackage{parskip}					%Stop auto-indenting
\usepackage{import}						%Import other .tex files for structure



%% BEGIN DOCUMENT -----------------------------------------------------------------
\begin{document}

%% TITLE --------------------------------------------------------------------------

\begin{titlepage}
	\vspace*{1cm}
	\Large
    \textit{\Institute}
    \vspace{0.25cm}

	\Huge
    \textbf{\Title}
    \vspace{0.25cm}

	\Large
	\Author
	\vspace{1.0cm}

	\normalsize
	\rule{\textwidth}{0.4pt}
	\newtheorem*{summary}{Summary}
	\begin{summary}
		\Docsummary
	\end{summary}
	\setcounter{tocdepth}{1}
	\tableofcontents
	\vfill
	\rule{\textwidth}{0.4pt}
\end{titlepage}

%% MAIN --------------------------------------------------------------------------

\section{Introduction and motivation}
\label{s:intro}
After writing for 3+ years in \LaTeXe, I still found myself often jumping inbetween projects to copy certain "complex" pieces of code that I needed, like fancy looking code blocks that I used once in an older project or some obscure \texttt{usepackage} I need. Thus, building a clearly documented template with custom environments and commands would solve this issue; everything I need while writing is now instantly at my fingertips!

It is highly recommended to compile documents made with this template with
\hologo{LuaLaTeX} to avoid the malfunctioning of certain \texttt{usepackages}.
\hologo{pdfLaTeX} is quite primitive on its own anyways. Luckily,
\hologo{TeX}live-full comes with \hologo{LuaLaTeX} out of the box. Furthermore,
\textit{Overleaf} works on \hologo{TeX}live, so with a simple setting switch
you can use \hologo{LuaLaTeX} there, too!

\newpage
\section{Commands}
\label{s:commands}
The template comes with a set of new useful commands.
\subsection{Document property variables}
\label{ss:docvars}
A few \textit{document variables} are available, embedded in the output PDF as metadata, as well as used to fill certain fields in the template.
\begin{itemize}
	\item \verb!\Title! prints the title of the document.
	\item \verb!\Author! prints the author(s) of the document.
	\item \verb!\Institute! prints the institute associated with the document, like a university or company.
	\item \verb!\Date! prints the date related to the document, like a deadline for a report.
	\item \verb!\Docsummary! prints the summary of the document.
\end{itemize}

All these variables are declared at the top of the template file, and are supposed to be altered as desired.

\subsection{The \texttt{now} command}
\label{ss:now}
The \verb|\now| command prints the current date and time using \verb|\datetime| and \verb|\today|. This document is compiled at \verb|\now|: \now.

\newpage
\section{Environments}
\label{s:envs}
The template defines some environments, that safe some precious time of setting up styles and nesting environments.

\subsection{The \texttt{code} environment}
\label{ss:code}
The \verb|code| environment renders a nice box with title and label to throw code in. For now, it does \textbf{not} support to import code from other files; it expects code to be nested in the environment. The environment has three arguments:

\begin{verbatim}
\begin{code}{<title>}{<language>}{<label>}
<code>
\end{code}
\end{verbatim}.



\newtheorem{exmp}{Example}[section]
\begin{exmp}
Imagine you are writing a tutorial on \textit{GNU Octave}, the free and open source alternative to \textit{MATLAB} by the GNU Foundation. You want to embed a Octave script to show how to plot two functions, $f(x) = x^{2}$ and $g(x) = \sin{(\cos{x})}$, in a figure for a fixed domain. You now only need to write:
\begin{verbatim}
	\begin{code}{Plotting two functions in GNU Octave}{octave}{lst:2func}
	clear vars;close all; clc;
	%% VARS
	x1 = -1:1e-3:1;
	x2 = -2:1e-3:2;
	%% FUNCS
	y1 = x1.^2;
	y2 = sin(cos(x2))
	%% PLOT
	figure(1)
	hold on; grid on;
	plot(x1,y1);
	plot(x2,y2);
	xlabel(x);
	ylabel(y);
	\end{code}
\end{verbatim}
When compiling this \LaTeXe code from the template, the output PDF renders
\begin{code}{Plotting two functions in GNU Octave}{octave}{lst:2func}
clear vars;close all; clc;
%% VARS
x1 = -1:1e-3:1;
x2 = -2:1e-3:2;
%% FUNCS
y1 = x1.^2;
y2 = sin(cos(x2))
%% PLOT
figure(1)
hold on; grid on;
plot(x1,y1);
plot(x2,y2);
xlabel(x);
ylabel(y);
\end{code},
which looks quite nice! Using \verb!\ref! we can refer to the code block using the label: \ref{lst:2func}.
\end{exmp}

Parameters other than the provided arguments can ofcourse be tuned manually in the template preamble itself.

\subsection{The \texttt{plot} environment}
\label{ss:plot}
The \verb|plot| environment is simply an alias to a \textit{PGFplot}
\texttt{axis} enviroment, which needs to be initiated inside a
\texttt{tikzpicture} evironment. This on its own does not allow as much
flexibility as a \texttt{figure} environment, so \verb|plot| also nests this
whole structure inside a figure with \verb|H| float. The syntax now becomes
very simple:

\begin{verbatim}
	\begin{plot}{<caption>}{<label>}
	[<pgfplotssset settings to add/override defaults>]
	<contents of the PGFplot axis object>
	\end{plot}
\end{verbatim}.

It is assumed one knows how to use PGFplots when reading this; if not, go to the \textit{Comprehensive \TeX{} Achive Network} for the full \href{https://mirror.koddos.net/CTAN/graphics/pgf/contrib/pgfplots/doc/pgfplots.pdf}{documentation}.

\begin{exmp}
Plots generated by GNU Octave look fine, but not stellar. The output of Listing
\ref{lst:2func} can look much nicer if we do the whole thing in PGFplots in
\LaTeXe{} natively. Let's use the \verb|plot| environment for that. Listing
\ref{lst:plotex} shows the code that plots the two functions. The plot is visible in \autoref{fig:2func}.

\begin{code}{PGFplot code to draw the two function in a subset of $\mathbb{R}^{2}$}{tex}{lst:plotex}
\begin{plot}{The two functions of Listing \ref{lst:2func}}{fig:2func}
	[clip=false,
	xmin=-2,
	xmax=+2,
	axis lines=middle,
	xlabel=$X$,
	ylabel=$Y$,
	grid style=dashed]
	\addplot[samples=50,domain=-1:1,blue]{x^2}
	node[right,pos=0.95]{ $f(x)=x^{2}$};
	\addplot[samples=50,domain=-2:2,red]{sin(deg(cos(deg(x))))}
	node[right,pos=0.95]{ $g(x)=\sin{(\cos{x})}$};
\end{plot}
\end{code}

\begin{plot}{The two functions of Listing \ref{lst:2func}}{fig:2func}
	[clip=false,
	xmin=-2,
	xmax=+2,
	axis lines=middle,
	xlabel=$X$,
	ylabel=$Y$,
	grid style=dashed]
	\addplot[samples=50,domain=-1:1,blue]{x^2}
	node[right,pos=0.95]{ $f(x)=x^{2}$};
	\addplot[samples=50,domain=-2:2,red]{sin(deg(cos(deg(x))))}
	node[right,pos=0.95]{ $g(x)=\sin{(\cos{x})}$};
\end{plot}

\end{exmp}

%% APPENDIX -----------------------------------------------------------------
\newpage
\begin{appendices}
\section{The template file}
The template file can be compiled as a \LaTeXe file as standalone and thus be used as a startpoint for any new projects. Below code functions as a offline reference to the template to see how things work, or just to grab some ideas from it. The full source can be found on the corresponding Github repository, found on this documents title page.
\import{./}{templatecodeblock.tex}


\end{appendices}

%% END DOCUMENT -----------------------------------------------------------------
\end{document}
%% ------------------------------------------------------------------------------
